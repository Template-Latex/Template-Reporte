% Template:     Template Reporte LaTeX
% Documento:    Funciones matemáticas
% Versión:      2.2.1 (26/05/2021)
% Codificación: UTF-8
%
% Autor: Pablo Pizarro R.
%        Facultad de Ciencias Físicas y Matemáticas
%        Universidad de Chile
%        pablo@ppizarror.com
%
% Sitio web:    [https://latex.ppizarror.com/reporte]
% Licencia MIT: [https://opensource.org/licenses/MIT]

\newcommand{\lpow}[2]{
	\ensuremath{{#1}_{#2}}
}
\newcommand{\pow}[2]{
	\ensuremath{{#1}^{#2}}
}
\newcommand{\aasin}[1][]{
	\ifx\hfuzz#1\hfuzz
		\ensuremath{\sin^{-1}#1}
	\else
		\ensuremath{{\sin}^{-1}}
	\fi
}
\newcommand{\aacos}[1][]{
	\ifx\hfuzz#1\hfuzz
		\ensuremath{\cos^{-1}#1}
	\else
		\ensuremath{\cos^{-1}}
	\fi
}
\newcommand{\aatan}[1][]{
	\ifx\hfuzz#1\hfuzz
		\ensuremath{\tan^{-1}#1}
	\else
		\ensuremath{\tan^{-1}}
	\fi
}
\newcommand{\aacsc}[1][]{
	\ifx\hfuzz#1\hfuzz
		\ensuremath{\csc^{-1}#1}
	\else
		\ensuremath{\csc^{-1}}
	\fi
}
\newcommand{\aasec}[1][]{
	\ifx\hfuzz#1\hfuzz
		\ensuremath{\sec^{-1}#1}
	\else
		\ensuremath{\sec^{-1}}
	\fi
}
\newcommand{\aacot}[1][]{
	\ifx\hfuzz#1\hfuzz
		\ensuremath{\cot^{-1}#1}
	\else
		\ensuremath{\cot^{-1}}
	\fi
}
\newcommand{\fracpartial}[2]{
	\ensuremath{\pdv{#1}{#2}}
}
\newcommand{\fracdpartial}[2]{
	\ensuremath{\pdv[2]{#1}{#2}}
}
\newcommand{\fracnpartial}[3]{
	\ensuremath{\pdv[#3]{#1}{#2}}
}
\newcommand{\fracderivat}[2]{
	\ensuremath{\dv{#1}{#2}}
}
\newcommand{\fracdderivat}[2]{
	\ensuremath{\dv[2]{#1}{#2}}
}
\newcommand{\fracnderivat}[3]{
	\ensuremath{\dv[#3]{#1}{#2}}
}
\newcommand{\topequal}[2]{
	\ensuremath{\overbrace{#1}^{\mathclap{#2}}}
}
\newcommand{\topequaltext}[2]{\topequal{#1}{\text{#2}}}
\newcommand{\underequal}[2]{
	\ensuremath{\underbrace{#1}_{\mathclap{#2}}}
}
\newcommand{\underequaltext}[2]{\underequal{#1}{\text{#2}}}
\newcommand{\topsequal}[2]{
	\ensuremath{\overbracket{#1}^{\mathclap{#2}}}
}
\newcommand{\topsequaltext}[2]{\topsequal{#1}{\text{#2}}}
\newcommand{\undersequal}[2]{
	\ensuremath{\underbracket{#1}_{\mathclap{#2}}}
}
\newcommand{\undersequaltext}[2]{\undersequal{#1}{\text{#2}}}
\newcommand{\floorexp}[1]{
	\ensuremath{\left\lfloor{#1}\right\rfloor}
}
\newcommand{\ceilexp}[1]{
	\ensuremath{\left\lceil{#1}\right\rceil}
}
\newcommand{\Mod}[1]{
	\ensuremath{\ (\mathrm{mod}\ #1)}
}
\newcommand{\bigp}[1]{
	\ensuremath{\big(#1\big)}
}
\newcommand{\biggp}[1]{
	\ensuremath{\bigg(#1\bigg)}
}
\newcommand{\bigc}[1]{
	\ensuremath{\big[#1\big]}
}
\newcommand{\biggc}[1]{
	\ensuremath{\bigg[#1\bigg]}
}
\newcommand{\bigb}[1]{
	\ensuremath{\big\{#1\big\}}
}
\newcommand{\biggb}[1]{
	\ensuremath{\bigg\{#1\bigg\}}
}
\newcommand{\divexp}{
	\ensuremath{\rm{div}\ }
}
\newcommand{\Autexp}{
	\ensuremath{\rm{Aut}}
}
\newcommand{\mathbit}[1]{
	\bm{#1}
}
\newcommand{\Diffexp}{
	\ensuremath{\rm{Diff}}
}
\newcommand{\Imexp}{
	\ensuremath{\rm{Im}}
}
\newcommand{\Imzexp}{
	\ensuremath{\rm{Im}(z)}
}
\newcommand{\Reexp}{
	\ensuremath{\rm{Re}}
}
\newcommand{\Rezexp}{
	\ensuremath{\rm{Re}(z)}
}
\newcommand{\overbar}[1]{\mkern 1.5mu\overline{\mkern-1.5mu#1\mkern-1.5mu}\mkern 1.5mu}
\makeatletter
\def\longtilde#1{\mathop{\vbox{\m@th\ialign{##\crcr\noalign{\kern3\p@}%
				\sortoftildefill\crcr\noalign{\kern3\p@\nointerlineskip}%
				$\hfil\displaystyle{#1}\hfil$\crcr}}}\limits}
\def\sortoftildefill{$\m@th \setbox\z@\hbox{$\braceld$}%
	\braceld\leaders\vrule \@height\ht\z@ \@depth\z@\hfill\braceru$}
\makeatother
\newcommand{\A}{\ensuremath{\mathcal{A}}}
\newcommand{\B}{\ensuremath{\mathcal{B}}}
\ifthenelse{\isundefined{\C}}{\newcommand{\C}{C}}{\let\oldC=\C}
\renewcommand{\C}{\ensuremath{\mathbb{C}}}
\newcommand{\D}{\ensuremath{\mathbb{D}}}
\newcommand{\E}{\ensuremath{\mathbb{E}}}
\newcommand{\F}{\ensuremath{\mathcal{F}}}
\ifthenelse{\isundefined{\G}}{\newcommand{\G}{G}}{\let\oldG=\G}
\renewcommand{\G}{\ensuremath{\mathcal{G}}}
\ifthenelse{\isundefined{\H}}{\newcommand{\H}{H}}{\let\oldH=\H}
\renewcommand{\H}{\ensuremath{\mathcal{H}}}
\newcommand{\I}{\ensuremath{\mathbb{I}}}
\newcommand{\J}{\ensuremath{\mathcal{J}}}
\newcommand{\K}{\ensuremath{\mathcal{K}}}
\let\oldL=\L
\renewcommand{\L}{\ensuremath{\mathcal{L}}}
\newcommand{\M}{\ensuremath{\mathcal{M}}}
\newcommand{\N}{\ensuremath{\mathbb{N}}}
\let\oldP=\P
\renewcommand{\P}{\ensuremath{\mathbb{P}}}
\newcommand{\Q}{\ensuremath{\mathbb{Q}}}
\newcommand{\R}{\ensuremath{\mathbb{R}}}
\let\oldS=\S
\renewcommand{\S}{\ensuremath{\mathcal{S}}}
\newcommand{\T}{\ensuremath{\mathcal{T}}}
\ifthenelse{\isundefined{\U}}{\newcommand{\U}{U}}{\let\oldU=\U}
\renewcommand{\U}{\ensuremath{\mathcal{U}}}
\newcommand{\V}{\ensuremath{\mathcal{V}}}
\newcommand{\W}{\ensuremath{\mathcal{W}}}
\newcommand{\X}{\ensuremath{\mathcal{X}}}
\newcommand{\Y}{\ensuremath{\mathcal{Y}}}
\newcommand{\Z}{\ensuremath{\mathbb{Z}}}
\newcommand{\asteq}{\ensuremath{\mathrel{{*}{=}}}}
\newcommand{\cdoteq}{\ensuremath{\mathrel{{\cdot}{=}}}}
\newcommand{\diveq}{\ensuremath{\mathrel{{/}{=}}}}
\newcommand{\eqast}{\ensuremath{\mathrel{{=}{*}}}}
\newcommand{\eqcdot}{\ensuremath{\mathrel{{=}{\cdot}}}}
\newcommand{\eqdiv}{\ensuremath{\mathrel{{=}{/}}}}
\newcommand{\eqeq}{\ensuremath{\mathrel{{=}{=}}}}
\newcommand{\eqminus}{\ensuremath{\mathrel{{=}{-}}}}
\newcommand{\eqnot}{\ensuremath{\mathrel{{=}{!}}}}
\newcommand{\eqplus}{\ensuremath{\mathrel{{=}{+}}}}
\newcommand{\eqtimes}{\ensuremath{\mathrel{{=}{\times}}}}
\newcommand{\minuseq}{\ensuremath{\mathrel{{-}{=}}}}
\newcommand{\minusminus}{\ensuremath{\mathrel{{-}{-}}}}
\newcommand{\noteq}{\ensuremath{\mathrel{{!}{=}}}}
\newcommand{\pluseq}{\ensuremath{\mathrel{{+}{=}}}}
\newcommand{\plusplus}{\ensuremath{\mathrel{{+}{+}}}}
\newcommand{\timeseq}{\ensuremath{\mathrel{{\times}{=}}}}
\makeatletter
	\renewenvironment{proof}[1][\proofname] {\par\pushQED{\qed}\normalfont\topsep6\p@\@plus6\p@\relax\trivlist\item[\hskip\labelsep\scshape\footnotesize#1\@addpunct{.}]\ignorespaces}{\popQED\endtrivlist\@endpefalse}
\makeatother
