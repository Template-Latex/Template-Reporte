% Template:     Template Reporte LaTeX
% Documento:    Configuración inicial del template
% Versión:      2.3.7 (21/08/2021)
% Codificación: UTF-8
%
% Autor: Pablo Pizarro R.
%        Facultad de Ciencias Físicas y Matemáticas
%        Universidad de Chile
%        pablo@ppizarror.com
%
% Sitio web:    [https://latex.ppizarror.com/reporte]
% Licencia MIT: [https://opensource.org/licenses/MIT]

% Se revisa si las variables no han sido borradas
\def\titulodelinforme {\titulodelreporte}
\def\subtituloinforme {\subtituloreporte}
\checkvardefined{\autordeldocumento}
\checkvardefined{\departamentouniversidad}
\checkvardefined{\localizacionuniversidad}
\checkvardefined{\nombredelcurso}
\checkvardefined{\nombrefacultad}
\checkvardefined{\nombreuniversidad}
\checkvardefined{\temaatratar}
\checkvardefined{\titulodelinforme}

% -----------------------------------------------------------------------------
% Se añade \xspace a las variables
% -----------------------------------------------------------------------------
\makeatletter
	\g@addto@macro\autordeldocumento\xspace
	\g@addto@macro\codigodelcurso\xspace
	\g@addto@macro\departamentouniversidad\xspace
	\g@addto@macro\localizacionuniversidad\xspace
	\g@addto@macro\nombredelcurso\xspace
	\g@addto@macro\nombrefacultad\xspace
	\g@addto@macro\nombreuniversidad\xspace
	\g@addto@macro\temaatratar\xspace
	\g@addto@macro\titulodelinforme\xspace
\makeatother

\ifthenelse{\equal{\subtituloinforme}{}}{
	\def\tituloinformehf {\titulodelinforme}
}{
	\def\tituloinformehf {\subtituloinforme}
}

% -----------------------------------------------------------------------------
% Verifica tamaños de fuentes
% -----------------------------------------------------------------------------
\corecheckfontsize{\captionfontsize}
\corecheckfontsize{\subcaptionfsize}

% -----------------------------------------------------------------------------
% Se activan números en menú marcadores del pdf
% -----------------------------------------------------------------------------
\ifthenelse{\equal{\cfgpdfsecnumbookmarks}{true}}{
	\bookmarksetup{numbered}}{
}

% -----------------------------------------------------------------------------
% Se define metadata del pdf
% -----------------------------------------------------------------------------
\ifthenelse{\equal{\cfgshowbookmarkmenu}{true}}{
	\def\cfgpdfpagemode {UseOutlines}
	}{
	\def\cfgpdfpagemode {UseNone}
}
\ifthenelse{\equal{\usepdfmetadata}{true}}{
	\def\pdfmetainfoautor {\autordeldocumento}
	\def\pdfmetainfocodigodelcurso {\codigodelcurso}
	\def\pdfmetainfonombredelcurso {\nombredelcurso}
	\def\pdfmetainfotema {\temaatratar}
	\def\pdfmetainfotitulo {\titulodelreporte}
	\def\pdfmetainfounidepto {\departamentouniversidad}
	\def\pdfmetainfounifacultad {\nombrefacultad}
	\def\pdfmetainfouninombre {\nombreuniversidad}
	\def\pdfmetainfouniubicacion {\localizacionuniversidad}
	\author{\pdfmetainfoautor}
	\title{\pdfmetainfotitulo}
}{
	\def\pdfmetainfoautor {}
	\def\pdfmetainfocodigodelcurso {}
	\def\pdfmetainfonombredelcurso {}
	\def\pdfmetainfotema {}
	\def\pdfmetainfotitulo {}
	\def\pdfmetainfounidepto {}
	\def\pdfmetainfounifacultad {}
	\def\pdfmetainfouninombre {}
	\def\pdfmetainfouniubicacion {}
}
\hypersetup{
	keeppdfinfo,
	bookmarksopen={\cfgpdfbookmarkopen},
	bookmarksopenlevel={\cfgbookmarksopenlevel},
	bookmarkstype={toc},
	pdfauthor={\pdfmetainfoautor},
	pdfcenterwindow={\cfgpdfcenterwindow},
	pdfcopyright={\cfgpdfcopyright},
	pdfcreator={LaTeX},
	pdfdisplaydoctitle={\cfgpdfdisplaydoctitle},
	pdfencoding={unicode},
	pdffitwindow={\cfgpdffitwindow},
	pdfinfo={
		Curso.Codigo={\pdfmetainfocodigodelcurso},
		Curso.Nombre={\pdfmetainfonombredelcurso},
		Documento.Autor={\pdfmetainfoautor},
		Documento.Tema={\pdfmetainfotema},
		Documento.Titulo={\pdfmetainfotitulo},
		Template.Autor.Alias={ppizarror},
		Template.Autor.Email={pablo@ppizarror.com},
		Template.Autor.Nombre={Pablo Pizarro R.},
		Template.Autor.Web={https://ppizarror.com},
		Template.Codificacion={UTF-8},
		Template.Fecha={21/08/2021},
		Template.Latex.Compilador={pdflatex},
		Template.Licencia.Tipo={MIT},
		Template.Licencia.Web={https://opensource.org/licenses/MIT},
		Template.Nombre={Template-Reporte},
		Template.Tipo={Normal},
		Template.Version.Dev={2.3.7-REPT},
		Template.Version.Hash={7FDFBEEA7BA4A607ACCFCB6FFE6B572B},
		Template.Version.Release={2.3.7},
		Template.Web.Dev={https://github.com/Template-Latex/Template-Reporte},
		Template.Web.Manual={https://latex.ppizarror.com/reporte},
		Universidad.Departamento={\pdfmetainfounidepto},
		Universidad.Facultad={\pdfmetainfounifacultad},
		Universidad.Nombre={\pdfmetainfouninombre},
		Universidad.Ubicacion={\pdfmetainfouniubicacion}
	},
	pdfkeywords={\cfgpdfkeywords},
	pdfmenubar={\cfgpdfmenubar},
	pdfpagelayout={\cfgpdflayout},
	pdfpagemode={\cfgpdfpagemode},
	pdfproducer={Template-Reporte v2.3.7 | (Pablo Pizarro R.) ppizarror.com},
	pdfremotestartview={Fit},
	pdfstartpage={1},
	pdfstartview={\cfgpdfpageview},
	pdfsubject={\pdfmetainfotema},
	pdftitle={\pdfmetainfotitulo},
	pdftoolbar={\cfgpdftoolbar}
}

% -----------------------------------------------------------------------------
% Establece la carpeta de imágenes por defecto
% -----------------------------------------------------------------------------
\graphicspath{{./\defaultimagefolder}}

% -----------------------------------------------------------------------------
% Elimina el espacio vertical de los flotantes
% -----------------------------------------------------------------------------
\makeatletter
\ifthenelse{\equal{\fpremovetopbottomcenter}{true}}{
	\setlength{\@fptop}{0pt}
	\setlength{\@fpbot}{0pt}
}{}
\makeatother

% -----------------------------------------------------------------------------
% Definición de valores e dimensiones
% -----------------------------------------------------------------------------
\renewcommand{\baselinestretch}{\documentinterline} % Ajuste del entrelineado
\setlength{\headheight}{64 pt} % Tamaño de la cabecera sin fancyhdr
% \setcounter{MaxMatrixCols}{20} % Número máximo de columnas en matrices
\setlength{\footnotemargin}{\marginfootnote pt} % Margen del footnote
\setlength{\columnsep}{\columnsepwidth em} % Separación entre columnas
\ifthenelse{\equal{\showlinenumbers}{true}}{
	\setlength{\linenumbersep}{0.50cm}
	\renewcommand\linenumberfont{\normalfont\tiny\color{\linenumbercolor}}
	}{
}

% -----------------------------------------------------------------------------
% Posición inicial de los objetos
% -----------------------------------------------------------------------------
\floatplacement{figure}{\imagedefaultplacement}
\floatplacement{table}{\tabledefaultplacement}
\floatplacement{tikz}{\tikzdefaultplacement}

% -----------------------------------------------------------------------------
% Configuración de los colores
% -----------------------------------------------------------------------------
\color{\maintextcolor} % Color principal
\arrayrulecolor{\tablelinecolor} % Color de las líneas de las tablas
\sethlcolor{\highlightcolor} % Color del subrayado por defecto
\ifthenelse{\equal{\showborderonlinks}{true}}{
	% Color de links con borde
	\hypersetup{
		citebordercolor=\numcitecolor,
		linkbordercolor=\linkcolor,
		urlbordercolor=\urlcolor
	}
}{
	% Color de links sin borde
	\hypersetup{ % No reorganizar
		hidelinks,
		colorlinks=true,
		citecolor=\numcitecolor,
		linkcolor=\linkcolor,
		urlcolor=\urlcolor
	}
}
\ifthenelse{\equal{\colorpage}{white}}{}{
	\pagecolor{\colorpage}
}

% -----------------------------------------------------------------------------
% Configuración de las leyendas
% -----------------------------------------------------------------------------
% Márgenes de las leyendas por defecto
\setcaptionmargincm{\captionlrmargin}
\ifthenelse{\equal{\captiontextbold}{true}}{ % Texto en negrita en etiquetas
	\renewcommand{\captiontextbold}{bf}}{
	\renewcommand{\captiontextbold}{}
}
\ifthenelse{\equal{\captiontextsubnumbold}{true}}{ % Número en negritas
	\renewcommand{\captiontextsubnumbold}{bf}}{
	\renewcommand{\captiontextsubnumbold}{}
}

% Se configura el texto de los caption
\captionsetup{
	font={\captionfontsize},
	labelfont={color=\captioncolor, \captiontextbold},
	labelformat={\captionlabelformat},
	labelsep={\captionlabelsep},
	textfont={color=\captiontextcolor},
	singlelinecheck=on
}

% Configura texto de los subcaption
\captionsetup*[subfigure]{
	font={\subcaptionfsize},
	labelfont={color=\captioncolor, \captiontextsubnumbold},
	labelformat={\subcaptionlabelformat},
	labelsep={\subcaptionlabelsep},
	% listofformat={subparens},
	lofdepth=1,
	textfont={color=\captiontextcolor},
	singlelinecheck=on
}
\captionsetup*[subtable]{
	font={\subcaptionfsize},
	labelfont={color=\captioncolor, \captiontextsubnumbold},
	labelformat={\subcaptionlabelformat},
	labelsep={\subcaptionlabelsep},
	% listofformat={subparens},
	lofdepth=1,
	textfont={color=\captiontextcolor},
	singlelinecheck=on
}

\makeatletter
\renewcommand\p@subfigure{\thefigure\captionsubchar}
\renewcommand\p@subtable{\thetable\captionsubchar}
\makeatother

% Configuración de márgenes en las figuras
\floatsetup[figure]{
	captionskip=\captiontbmarginfigure pt
}

% Configuración de márgenes en las tablas
\floatsetup[table]{
	captionskip=\captiontbmargintable pt
}

% Caption superior en figuras
\ifthenelse{\equal{\figurecaptiontop}{true}}{
	\floatsetup[figure]{position=above}}{
}

% Caption superior en tablas
\ifthenelse{\equal{\tablecaptiontop}{true}}{
	\floatsetup[table]{position=top}
	}{
	\floatsetup[table]{position=bottom}
}

% Alineado de leyendas
\ifthenelse{\equal{\captionalignment}{justified}}{ % Leyenda justificada
	\captionsetup{
		format=plain,
		justification=justified
	}
}{
\ifthenelse{\equal{\captionalignment}{centered}}{ % Leyenda centrada
	\captionsetup{
		justification=centering
	}
}{
\ifthenelse{\equal{\captionalignment}{left}}{ % Leyenda alineada a la izquierda
	\captionsetup{
		justification=raggedright,
		singlelinecheck=false
	}
}{
\ifthenelse{\equal{\captionalignment}{right}}{ % Leyenda alineada a la derecha
	\captionsetup{
		justification=raggedleft,
		singlelinecheck=false
	}
}{
	\throwbadconfig{Posicion de leyendas desconocida}{\captionalignment}{justified,centered,left,right}}}}
}

% -----------------------------------------------------------------------------
% Configuración de referencias y citas
% -----------------------------------------------------------------------------
\ifthenelse{\equal{\stylecitereferences}{natbib}}{
	\def\twocolumnreferencesmargin{-0.35cm}
	\bibliographystyle{\natbibrefstyle}
	\setlength{\bibsep}{\natbibrefsep pt}
	\newcommand{\shortcite}[1]{\citep{#1}}
	\newcommand{\fullcite}[1]{\citet{#1}}
	% Caracteres citas
	\setcitestyle{open={\natbibrefcitecharopen},close={\natbibrefcitecharclose}}
	% Separador citas
	\ifthenelse{\equal{\natbibrefcitesepcomma}{true}}{
		\setcitestyle{comma}
	}{
		\setcitestyle{semicolon}
	}
	% Tipo citas
	\ifthenelse{\equal{\natbibrefcitetype}{numbers}}{
		\setcitestyle{numbers}
	}{
	\ifthenelse{\equal{\natbibrefcitetype}{authoryear}}{
		\setcitestyle{authoryear}
	}{
	\ifthenelse{\equal{\natbibrefcitetype}{super}}{
		\setcitestyle{super}
	}{
		\throwbadconfig{Tipo cita natbib desconocido}{\natbibrefcitetype}{numbers,authoryear,super}}}
	}
}{
\ifthenelse{\equal{\stylecitereferences}{apacite}}{
	\def\twocolumnreferencesmargin{-0.39cm}
	\bibliographystyle{\apacitestyle}
	\setlength{\bibitemsep}{\apaciterefsep pt}
	\newcommand{\citep}[1]{\fullcite{#1}}
	\newcommand{\citet}[1]{\shortcite{#1}}
}{
\ifthenelse{\equal{\stylecitereferences}{bibtex}}{
	\def\twocolumnreferencesmargin{-0.35cm}
	\bibliographystyle{\bibtexstyle}
	\newlength{\bibitemsep}
	\setlength{\bibitemsep}{.2\baselineskip plus .05\baselineskip minus .05\baselineskip}
	\newlength{\bibparskip}\setlength{\bibparskip}{0pt}
	\let\oldthebibliography\thebibliography
	\renewcommand\thebibliography[1]{
		\oldthebibliography{#1}
		\setlength{\parskip}{\bibitemsep}
		\setlength{\itemsep}{\bibparskip}
	}
	\setlength{\bibitemsep}{\bibtexrefsep pt}
}{
\ifthenelse{\equal{\stylecitereferences}{custom}}{
	\coretemplatemessage{Usando estilo citas referencias custom, importar librerias y configuraciones posterior al llamado de template.tex en archivo principal}
}{
	\throwbadconfig{Estilo citas desconocido}{\stylecitereferences}{bibtex,apacite,natbib,custom}}}}
}

% Crea referencias enumeradas en apacite
\makeatletter
\ifthenelse{\equal{\stylecitereferences}{apacite}}{
	\ifthenelse{\equal{\apaciterefnumber}{true}}{
		\newcounter{apaciteNumberCounter}
		\renewcommand{\theapaciteNumberCounter}{% Formato de número
			\apaciterefcitecharopen\arabic{apaciteNumberCounter}\apaciterefcitecharclose
		}
		\patchcmd{\@lbibitem}{\item[}{\item[\stepcounter{apaciteNumberCounter}{\hss\llap{\theapaciteNumberCounter}\quad}}{}{}
		\setlength{\bibleftmargin}{2.54em}
		\setlength{\bibindent}{-0.54em}
	}{}
}{}
\makeatother

% Desactiva la URL de apacite
\ifthenelse{\equal{\stylecitereferences}{apacite}}{
	\ifthenelse{\equal{\apaciteshowurl}{false}}{
		\renewenvironment{APACrefURL}[1][]{}{}
		\AtBeginEnvironment{APACrefURL}{\renewcommand{\url}[1]{}}
		\renewcommand{\doiprefix}{doi:~\kern-1pt}
	}{}
}{}

% Referencias en 2 columnas
\makeatletter
\ifthenelse{\equal{\twocolumnreferences}{true}}{
	\renewenvironment{thebibliography}[1]
	{\begin{multicols}{2}[\section*{\refname}\vspace{\twocolumnreferencesmargin}]
		\@mkboth{\MakeUppercase\refname}{\MakeUppercase\refname}
		\list{\@biblabel{\@arabic\c@enumiv}}
		{\settowidth\labelwidth{\@biblabel{#1}}
			\leftmargin\labelwidth
			\advance\leftmargin\labelsep
			\@openbib@code
			\usecounter{enumiv}
			\let\p@enumiv\@empty
			\renewcommand\theenumiv{\@arabic\c@enumiv}}
		\sloppy
		\clubpenalty 4000
		\@clubpenalty \clubpenalty
		\widowpenalty 4000
		\sfcode`\.\@m}
		{\def\@noitemerr
		{\@latex@warning{Ambiente `thebibliography' no definido}}
		\endlist\end{multicols}}}{}
\makeatother

% -----------------------------------------------------------------------------
% Configuración anexo
% -----------------------------------------------------------------------------
\patchcmd{\appendices}{\quad}{\sectionappendixlastchar\quad}{}{}

% -----------------------------------------------------------------------------
% Se añade listings (código fuente) a tocloft
% -----------------------------------------------------------------------------
\begingroup
	\makeatletter
	\let\newcounter\@gobble\let\setcounter\@gobbletwo
	\globaldefs\@ne\let\c@loldepth\@ne
	\newlistof{listings}{lol}{\lstlistlistingname}
	\newlistentry{lstlisting}{lol}{0}
	\makeatother
\endgroup

% -----------------------------------------------------------------------------
% Crea índice de ecuaciones
% -----------------------------------------------------------------------------
\newcommand{\listindexequationsname}{\nomlteqn}
\newlistof{myindexequations}{equ}{\listindexequationsname}
\newcommand{\myindexequations}[1]{
	\addcontentsline{equ}{myindexequations}{\protect\numberline{\theequation}#1}
}
\setcounter{templateIndexEquations}{0}
\DeclareTotalCounter{templateIndexEquations}

% -----------------------------------------------------------------------------
% Reconfiguración de tamaño de páginas
% -----------------------------------------------------------------------------
\makeatletter
	\def\ifGm@preamble#1{\@firstofone}
	\appto\restoregeometry{
		\pdfpagewidth=\paperwidth
		\pdfpageheight=\paperheight}
	\apptocmd\newgeometry{
		\pdfpagewidth=\paperwidth
		\pdfpageheight=\paperheight}{}{}
\makeatother

% -----------------------------------------------------------------------------
% Configuración de hbox y vbox
% -----------------------------------------------------------------------------
\hfuzz=200pt
\vfuzz=200pt
\hbadness=\maxdimen
\vbadness=\maxdimen

% -----------------------------------------------------------------------------
% Configura las fuentes
% -----------------------------------------------------------------------------
\makeatletter
\def\Hv@scale {.95}
\makeatother

% -----------------------------------------------------------------------------
% Configuraciones de las tablas
% -----------------------------------------------------------------------------
\makeatletter % Reinicia el número de cada fila en todas las tablas
\preto\tabular{\global\rownum=\z@}
\preto\tabularx{\global\rownum=\z@}
\makeatother

% -----------------------------------------------------------------------------
% Se activa el modo estricto de revisión de números de página
% -----------------------------------------------------------------------------
\strictpagecheck

% -----------------------------------------------------------------------------
% Actualización margen títulos
% -----------------------------------------------------------------------------
\titlespacing{\section}{0pt}{20pt}{10pt}
\titlespacing{\subsection}{0pt}{15pt}{10pt}

% -----------------------------------------------------------------------------
% Se activa el word-wrap para textos con \texttt{}
% -----------------------------------------------------------------------------
\ttfamily \hyphenchar\the\font=`\-
\expandafter\def\expandafter\UrlBreaks\expandafter{\UrlBreaks%
	\do\a\do\b\do\c\do\d\do\e\do\f\do\g\do\h\do\i\do\j%
	\do\k\do\l\do\m\do\n\do\o\do\p\do\q\do\r\do\s\do\t%
	\do\u\do\v\do\w\do\x\do\y\do\z\do\A\do\B\do\C\do\D%
	\do\E\do\F\do\G\do\H\do\I\do\J\do\K\do\L\do\M\do\N%
	\do\O\do\P\do\Q\do\R\do\S\do\T\do\U\do\V\do\W\do\X%
	\do\Y\do\Z%
}

% -----------------------------------------------------------------------------
% Se define el tipo de texto de los url
% -----------------------------------------------------------------------------
\urlstyle{\fonturl}

% -----------------------------------------------------------------------------
% Configuraciones del motor de compilación
% -----------------------------------------------------------------------------
\ifthenelse{\equal{\compilertype}{pdf2latex}}{
	% Nivel de compresión
	\pdfcompresslevel=\pdfcompilecompression
	
	% El óptimo es 2, según
	% https://texdoc.org/serve/pdftex-a.pdf/0 p.20
	\pdfdecimaldigits=2
	
	% Inclusión de PDF
	\pdfinclusionerrorlevel=0
	
	% Versión
	\pdfminorversion=\pdfcompileversion
	
	% Compresión de objetos
	\pdfobjcompresslevel=\pdfcompileobjcompression
}{
\ifthenelse{\equal{\compilertype}{xelatex}}{
}{
\ifthenelse{\equal{\compilertype}{lualatex}}{
}{
	\throwbadconfig{Compilador desconocido}{\compilertype}{pdf2latex,xelatex,lualatex}}}
}

% -----------------------------------------------------------------------------
% Crea las sub-sub-sub-secciones
% -----------------------------------------------------------------------------
\newcounter{subsubsubsection}[subsubsection]
\ifthenelse{\equal{\showdotaftersnum}{true}}{ % Agrega punto tras el número
	\renewcommand{\thesubsubsubsection}{\thesubsubsection.\arabic{subsubsubsection}.}
	\renewcommand{\theparagraph}{\thesubsubsubsection.\arabic{paragraph}.}
}{
	\renewcommand{\thesubsubsubsection}{\thesubsubsection.\arabic{subsubsubsection}}
	\renewcommand{\theparagraph}{\thesubsubsubsection.\arabic{paragraph}}
}

% Límite máximo profundidad
\setcounter{secnumdepth}{4}

% Agrega compatibilidad de sub-sub-sub-secciones al TOC
\makeatletter
	\def\toclevel@subsubsubsection {4}
	\def\toclevel@paragraph {5}
	\def\toclevel@subparagraph {6}
	\renewcommand\paragraph{\@startsection{paragraph}{5}{\z@}
		{3.25ex \@plus 1ex \@minus .2ex}
		{-1em}
		{\normalfont\normalsize\bfseries}}
	\renewcommand\subparagraph{\@startsection{subparagraph}{6}{\parindent}
		{3.25ex \@plus 1ex \@minus .2ex}
		{-1em}
		{\normalfont\normalsize\bfseries}}
	\ifthenelse{\equal{\showdotaftersnum}{true}}{
		\def\l@subsubsubsection {\@dottedtocline{4}{7.83em}{4.15em}} % Incremento 0.77+3.35 a 3.35
		\def\l@paragraph {\@dottedtocline{5}{11.98em}{4.92em}}
		\def\l@subparagraph {\@dottedtocline{6}{14.65em}{5.69em}}
	}{
		\def\l@subsubsubsection {\@dottedtocline{4}{6.97em}{4em}}
		\def\l@paragraph {\@dottedtocline{5}{10.97em}{5em}}
		\def\l@subparagraph {\@dottedtocline{6}{14em}{6em}}
	}
\makeatother

% -----------------------------------------------------------------------------
% Profundidad del índice y bookmarks pdf
% -----------------------------------------------------------------------------
\setcounter{tocdepth}{\indexdepth}

% -----------------------------------------------------------------------------
% Restauración número footnote
% -----------------------------------------------------------------------------
\ifthenelse{\equal{\footnoterestart}{none}}{
	% \counterwithout*{footnote}{chapter}
}{
\ifthenelse{\equal{\footnoterestart}{sec}}{
	\counterwithin*{footnote}{section}
}{
\ifthenelse{\equal{\footnoterestart}{ssec}}{
	\counterwithin*{footnote}{subsection}
}{
\ifthenelse{\equal{\footnoterestart}{sssec}}{
	\counterwithin*{footnote}{subsubsection}
}{
\ifthenelse{\equal{\footnoterestart}{ssssec}}{
	\counterwithin*{footnote}{subsubsubsection}
}{
\ifthenelse{\equal{\footnoterestart}{page}}{
	\counterwithin*{footnote}{page}
}{
\ifthenelse{\equal{\footnoterestart}{chap}}{
	\counterwithin*{footnote}{chapter}
}{
	\throwbadconfig{Formato reinicio numero footnote desconocido}{\footnoterestart}{none,chap,page,sec,ssec,sssec,ssssec}}}}}}}
}

% -----------------------------------------------------------------------------
% Restauración número ecuación, NOTA: NO hace nada, sólo se modifica en title.tex
% -----------------------------------------------------------------------------
\ifthenelse{\equal{\equationrestart}{none}}{
}{
\ifthenelse{\equal{\equationrestart}{chap}}{
}{
\ifthenelse{\equal{\equationrestart}{sec}}{
}{
\ifthenelse{\equal{\equationrestart}{ssec}}{
}{
\ifthenelse{\equal{\equationrestart}{sssec}}{
}{
\ifthenelse{\equal{\equationrestart}{ssssec}}{
}{
	\throwbadconfig{Formato reinicio numero ecuacion desconocido}{\equationrestart}{none,chap,sec,ssec,sssec,ssssec}}}}}}
}

% -----------------------------------------------------------------------------
% Configuración elementos matemáticos
% -----------------------------------------------------------------------------
\newtheoremstyle{miestilo}{\baselineskip}{3pt}{\itshape}{}{\bfseries}{}{.5em}{}
\newtheoremstyle{miobs}{\baselineskip}{3pt}{}{}{\bfseries}{}{.5em}{}
\theoremstyle{miestilo}

% Configura números
\ifthenelse{\equal{\showsectioncaptionmat}{none}}{
	\newtheorem{defn}{\namemathdefn}
	\newtheorem{teo}{\namemaththeorem}
	\newtheorem{cor}{\namemathcol}
	\newtheorem{lema}{\namemathlem}
	\newtheorem{prop}{\namemathprp}
}{
\ifthenelse{\equal{\showsectioncaptionmat}{chap}}{
	\newtheorem{defn}{\namemathdefn}[chapter]
	\newtheorem{teo}{\namemaththeorem}[chapter]
	\newtheorem{cor}{\namemathcol}[chapter]
	\newtheorem{lema}{\namemathlem}[chapter]
	\newtheorem{prop}{\namemathprp}[chapter]
}{
\ifthenelse{\equal{\showsectioncaptionmat}{sec}}{
	\newtheorem{defn}{\namemathdefn}[section]
	\newtheorem{teo}{\namemaththeorem}[section]
	\newtheorem{cor}{\namemathcol}[section]
	\newtheorem{lema}{\namemathlem}[section]
	\newtheorem{prop}{\namemathprp}[section]
}{
\ifthenelse{\equal{\showsectioncaptionmat}{ssec}}{
	\newtheorem{defn}{\namemathdefn}[subsection]
	\newtheorem{teo}{\namemaththeorem}[subsection]
	\newtheorem{cor}{\namemathcol}[subsection]
	\newtheorem{lema}{\namemathlem}[subsection]
	\newtheorem{prop}{\namemathprp}[subsection]
}{
\ifthenelse{\equal{\showsectioncaptionmat}{sssec}}{
	\newtheorem{defn}{\namemathdefn}[subsubsection]
	\newtheorem{teo}{\namemaththeorem}[subsubsection]
	\newtheorem{cor}{\namemathcol}[subsubsection]
	\newtheorem{lema}{\namemathlem}[subsubsection]
	\newtheorem{prop}{\namemathprp}[subsubsection]
}{
\ifthenelse{\equal{\showsectioncaptionmat}{ssssec}}{
	\newtheorem{defn}{\namemathdefn}[subsubsubsection]
	\newtheorem{teo}{\namemaththeorem}[subsubsubsection]
	\newtheorem{cor}{\namemathcol}[subsubsubsection]
	\newtheorem{lema}{\namemathlem}[subsubsubsection]
	\newtheorem{prop}{\namemathprp}[subsubsubsection]
}{
	\throwbadconfig{Valor configuracion incorrecto}{\showsectioncaptionmat}{none,chap,sec,ssec,sssec,ssssec}}}}}}
}
\theoremstyle{miobs}
\newtheorem*{ej}{\namemathej}
\newtheorem*{obs}{\namemathobs}

% -----------------------------------------------------------------------------
% Configuraciones del idioma
% -----------------------------------------------------------------------------
% Desactiva caracteres acentuados en operaciones matemáticas
\unaccentedoperators

% -----------------------------------------------------------------------------
% Configura número de objetos en el final del documento
% -----------------------------------------------------------------------------
\AtEndDocument{
	\addtocounter{equation}{\value{templateEquations}}
	\addtocounter{figure}{\value{templateFigures}}
	\addtocounter{lstlisting}{\value{templateListings}}
	\addtocounter{table}{\value{templateTables}}
}

% -----------------------------------------------------------------------------
% Formato de columnas
% -----------------------------------------------------------------------------
% Centrado
\newcolumntype{C}[1]{>{\centering\let\newline\\\arraybackslash\hspace{0pt}}m{#1}}
\newcolumntype{\CColor}[2]{>{\columncolor{#1}\centering\let\newline\\\arraybackslash\hspace{0pt}}m{#2}}

\newcolumntype{P}[1]{>{\centering\let\newline\\\arraybackslash\hspace{0pt}}p{#1}}
\newcolumntype{\PColor}[2]{>{\columncolor{#1}\centering\let\newline\\\arraybackslash\hspace{0pt}}p{#2}}

\newcolumntype{B}[1]{>{\centering\let\newline\\\arraybackslash\hspace{0pt}}b{#1}}
\newcolumntype{\BColor}[2]{>{\columncolor{#1}\centering\let\newline\\\arraybackslash\hspace{0pt}}b{#2}}

% Izquierda
\newcolumntype{L}[1]{>{\raggedright\let\newline\\\arraybackslash\hspace{0pt}}m{#1}}
\newcolumntype{\LColor}[2]{>{\columncolor{#1}\raggedright\let\newline\\\arraybackslash\hspace{0pt}}m{#2}}
\newcolumntype{T}[1]{>{\raggedright\let\newline\\\arraybackslash\hspace{0pt}}p{#1}}
\newcolumntype{\TColor}[2]{>{\columncolor{#1}\raggedright\let\newline\\\arraybackslash\hspace{0pt}}p{#2}}
\newcolumntype{F}[1]{>{\raggedright\let\newline\\\arraybackslash\hspace{0pt}}b{#1}}
\newcolumntype{\FColor}[2]{>{\columncolor{#1}\raggedright\let\newline\\\arraybackslash\hspace{0pt}}b{#2}}

% Derecha
\newcolumntype{R}[1]{>{\raggedleft\let\newline\\\arraybackslash\hspace{0pt}}m{#1}}
\newcolumntype{\RColor}[2]{>{\columncolor{#1}\raggedleft\let\newline\\\arraybackslash\hspace{0pt}}m{#2}}
\newcolumntype{H}[1]{>{\raggedleft\let\newline\\\arraybackslash\hspace{0pt}}p{#1}}
\newcolumntype{\HColor}[2]{>{\columncolor{#1}\raggedleft\let\newline\\\arraybackslash\hspace{0pt}}p{#2}}
\newcolumntype{G}[1]{>{\raggedleft\let\newline\\\arraybackslash\hspace{0pt}}b{#1}}
\newcolumntype{\GColor}[2]{>{\columncolor{#1}\raggedleft\let\newline\\\arraybackslash\hspace{0pt}}b{#2}}

% -----------------------------------------------------------------------------
% Parcha el entorno multicols
% -----------------------------------------------------------------------------
\let\SOURCEcaptionlrmargin\captionlrmargin
\BeforeBeginEnvironment{multicols}{%
	\def\captionlrmargin {\captionlrmarginmc}%
	\global\def\GLOBALenvmulticol {true}%
	\setcaptionmargincm{\captionlrmargin}%
}
\AfterEndEnvironment{multicols}{%
	\def\captionlrmargin {\SOURCEcaptionlrmargin}%
	\global\def\GLOBALenvmulticol {false}%
	\setcaptionmargincm{\captionlrmargin}%
}

% -----------------------------------------------------------------------------
% Configura estilos de listas
% -----------------------------------------------------------------------------
% Enumerate
\renewcommand{\labelenumi}{\senumerti}
\renewcommand{\labelenumii}{\senumertii}
\renewcommand{\labelenumiii}{\senumertiii}
\renewcommand{\labelenumiv}{\senumertiv}

% Itemize
\renewcommand{\labelitemi}{\sitemizei}
\renewcommand{\labelitemii}{\sitemizeii}
\renewcommand{\labelitemiii}{\sitemizeiii}
\renewcommand{\labelitemiv}{\sitemizeiv}

% -----------------------------------------------------------------------------
% Operaciones especiales Template-Reporte
% -----------------------------------------------------------------------------
\title{\titulodelinforme}
\author{\autordeldocumento}
\date{\fechadelreporte}
\def\predocpageromannumber {false}
\def\predocresetpagenumber {false}
\def\tablaintegrantes {}
\makeatletter
\def\inserttitle {
	\null
	\vskip \titlesupmargin
	\begin{center}
		\let \footnote \thanks
		{\titlefontsize \titlefontstyle \@title \par}
		\ifthenelse{\equal{\titleshowauthor}{true}}{
			\vskip \titlelinemargin
			{\normalsize
				\begin{tabular}[t]{c}
					\@author
				\end{tabular}\par}
		}{}
		\ifthenelse{\equal{\titleshowcourse}{true}}{
			\vskip 0.5em
			{\normalsize \codigodelcurso\ \hfpdashcharstyle\ \nombredelcurso}
		}{}
		\ifthenelse{\equal{\titleshowdate}{true}}{
			\vskip 1em
			{\normalsize \@date}
		}{}
	\end{center}
	\par
	\vskip \titlelinemargin}
\makeatother
