% Template:     Template Reporte LaTeX
% Documento:    Funciones matemáticas
% Versión:      1.1.2 (23/08/2019)
% Codificación: UTF-8
%
% Autor: Pablo Pizarro R. @ppizarror
%        Facultad de Ciencias Físicas y Matemáticas
%        Universidad de Chile
%        pablo.pizarro@ing.uchile.cl, ppizarror.com
%
% Sitio web:    [https://latex.ppizarror.com/Template-Reporte/]
% Licencia MIT: [https://opensource.org/licenses/MIT]

\newcommand{\lpow}[2]{
	\ensuremath{{#1}_{#2}}
}
\newcommand{\pow}[2]{
	\ensuremath{{#1}^{#2}}
}
\newcommand{\aasin}[1][]{
	\ifx\hfuzz#1\hfuzz
		\ensuremath{\sin^{-1}#1}
	\else
		\ensuremath{{\sin}^{-1}}
	\fi
}
\newcommand{\aacos}[1][]{
	\ifx\hfuzz#1\hfuzz
		\ensuremath{\cos^{-1}#1}
	\else
		\ensuremath{\cos^{-1}}
	\fi
}
\newcommand{\aatan}[1][]{
	\ifx\hfuzz#1\hfuzz
		\ensuremath{\tan^{-1}#1}
	\else
		\ensuremath{\tan^{-1}}
	\fi
}
\newcommand{\aacsc}[1][]{
	\ifx\hfuzz#1\hfuzz
		\ensuremath{\csc^{-1}#1}
	\else
		\ensuremath{\csc^{-1}}
	\fi
}
\newcommand{\aasec}[1][]{
	\ifx\hfuzz#1\hfuzz
		\ensuremath{\sec^{-1}#1}
	\else
		\ensuremath{\sec^{-1}}
	\fi
}
\newcommand{\aacot}[1][]{
	\ifx\hfuzz#1\hfuzz
		\ensuremath{\cot^{-1}#1}
	\else
		\ensuremath{\cot^{-1}}
	\fi
}
\newcommand{\fracpartial}[2]{
	\ensuremath{\pdv{#1}{#2}}
}
\newcommand{\fracdpartial}[2]{
	\ensuremath{\pdv[2]{#1}{#2}}
}
\newcommand{\fracnpartial}[3]{
	\ensuremath{\pdv[#3]{#1}{#2}}
}
\newcommand{\fracderivat}[2]{
	\ensuremath{\dv{#1}{#2}}
}
\newcommand{\fracdderivat}[2]{
	\ensuremath{\dv[2]{#1}{#2}}
}
\newcommand{\fracnderivat}[3]{
	\ensuremath{\dv[#3]{#1}{#2}}
}
\newcommand{\topequal}[2]{
	\ensuremath{\overbrace{#1}^{\mathclap{#2}}}
}
\newcommand{\underequal}[2]{
	\ensuremath{\underbrace{#1}_{\mathclap{#2}}}
}
\newcommand{\topsequal}[2]{
	\ensuremath{\overbracket{#1}^{\mathclap{#2}}}
}
\newcommand{\undersequal}[2]{
	\ensuremath{\underbracket{#1}_{\mathclap{#2}}}
}
\newcommand{\bigp}[1]{
	\ensuremath{\big(#1\big)}
}
\newcommand{\biggp}[1]{
	\ensuremath{\bigg(#1\bigg)}
}
\newcommand{\bigc}[1]{
	\ensuremath{\big[#1\big]}
}
\newcommand{\biggc}[1]{
	\ensuremath{\bigg[#1\bigg]}
}
\newcommand{\bigb}[1]{
	\ensuremath{\big\{#1\big\}}
}
\newcommand{\biggb}[1]{
	\ensuremath{\bigg\{#1\bigg\}}
}
\newcommand{\divexp}{
	\ensuremath{\rm{div}\ }
}
\newcommand{\Autexp}{
	\ensuremath{\rm{Aut}}
}
\newcommand{\Diffexp}{
	\ensuremath{\rm{Diff}}
}
\newcommand{\Imexp}{
	\ensuremath{\rm{Im}}
}
\newcommand{\Imzexp}{
	\ensuremath{\rm{Im}(z)}
}
\newcommand{\Reexp}{
	\ensuremath{\rm{Re}}
}
\newcommand{\Rezexp}{
	\ensuremath{\rm{Re}(z)}
}
\newcommand{\A}{\mathcal{A}}
\let\oldC=\C
\renewcommand{\C}{\mathbb{C}}
\newcommand{\D}{\mathbb{D}}
\newcommand{\E}{\mathbb{E}}
\newcommand{\F}{\mathcal{F}}
\let\oldG=\G
\renewcommand{\G}{\mathcal{G}}
\let\oldH=\H
\renewcommand{\H}{\mathcal{H}}
\newcommand{\K}{\mathcal{K}}
\let\oldL=\L
\renewcommand{\L}{\mathcal{L}}
\newcommand{\M}{\mathcal{M}}
\newcommand{\N}{\mathbb{N}}
\let\oldP=\P
\renewcommand{\P}{\mathbb{P}}
\newcommand{\Q}{\mathbb{Q}}
\newcommand{\R}{\mathbb{R}}
\let\oldS=\S
\renewcommand{\S}{\mathcal{S}}
\newcommand{\T}{\mathcal{T}}
\newcommand{\Z}{\mathbb{Z}}
\newcommand{\1}{\mathbf{1}}
\newcommand{\overbar}[1]{\mkern 1.5mu\overline{\mkern-1.5mu#1\mkern-1.5mu}\mkern 1.5mu}
